\PassOptionsToPackage{unicode=true}{hyperref} % options for packages loaded elsewhere
\PassOptionsToPackage{hyphens}{url}
%
\documentclass[]{article}
\usepackage{lmodern}
\usepackage{amssymb,amsmath}
\usepackage{ifxetex,ifluatex}
\usepackage{fixltx2e} % provides \textsubscript
\ifnum 0\ifxetex 1\fi\ifluatex 1\fi=0 % if pdftex
  \usepackage[T1]{fontenc}
  \usepackage[utf8]{inputenc}
  \usepackage{textcomp} % provides euro and other symbols
\else % if luatex or xelatex
  \usepackage{unicode-math}
  \defaultfontfeatures{Ligatures=TeX,Scale=MatchLowercase}
\fi
% use upquote if available, for straight quotes in verbatim environments
\IfFileExists{upquote.sty}{\usepackage{upquote}}{}
% use microtype if available
\IfFileExists{microtype.sty}{%
\usepackage[]{microtype}
\UseMicrotypeSet[protrusion]{basicmath} % disable protrusion for tt fonts
}{}
\IfFileExists{parskip.sty}{%
\usepackage{parskip}
}{% else
\setlength{\parindent}{0pt}
\setlength{\parskip}{6pt plus 2pt minus 1pt}
}
\usepackage{hyperref}
\hypersetup{
            pdftitle={Statistical Inference Course Project - Part I},
            pdfauthor={Evgeniia Golovina},
            pdfborder={0 0 0},
            breaklinks=true}
\urlstyle{same}  % don't use monospace font for urls
\usepackage[margin=1in]{geometry}
\usepackage{color}
\usepackage{fancyvrb}
\newcommand{\VerbBar}{|}
\newcommand{\VERB}{\Verb[commandchars=\\\{\}]}
\DefineVerbatimEnvironment{Highlighting}{Verbatim}{commandchars=\\\{\}}
% Add ',fontsize=\small' for more characters per line
\usepackage{framed}
\definecolor{shadecolor}{RGB}{248,248,248}
\newenvironment{Shaded}{\begin{snugshade}}{\end{snugshade}}
\newcommand{\AlertTok}[1]{\textcolor[rgb]{0.94,0.16,0.16}{#1}}
\newcommand{\AnnotationTok}[1]{\textcolor[rgb]{0.56,0.35,0.01}{\textbf{\textit{#1}}}}
\newcommand{\AttributeTok}[1]{\textcolor[rgb]{0.77,0.63,0.00}{#1}}
\newcommand{\BaseNTok}[1]{\textcolor[rgb]{0.00,0.00,0.81}{#1}}
\newcommand{\BuiltInTok}[1]{#1}
\newcommand{\CharTok}[1]{\textcolor[rgb]{0.31,0.60,0.02}{#1}}
\newcommand{\CommentTok}[1]{\textcolor[rgb]{0.56,0.35,0.01}{\textit{#1}}}
\newcommand{\CommentVarTok}[1]{\textcolor[rgb]{0.56,0.35,0.01}{\textbf{\textit{#1}}}}
\newcommand{\ConstantTok}[1]{\textcolor[rgb]{0.00,0.00,0.00}{#1}}
\newcommand{\ControlFlowTok}[1]{\textcolor[rgb]{0.13,0.29,0.53}{\textbf{#1}}}
\newcommand{\DataTypeTok}[1]{\textcolor[rgb]{0.13,0.29,0.53}{#1}}
\newcommand{\DecValTok}[1]{\textcolor[rgb]{0.00,0.00,0.81}{#1}}
\newcommand{\DocumentationTok}[1]{\textcolor[rgb]{0.56,0.35,0.01}{\textbf{\textit{#1}}}}
\newcommand{\ErrorTok}[1]{\textcolor[rgb]{0.64,0.00,0.00}{\textbf{#1}}}
\newcommand{\ExtensionTok}[1]{#1}
\newcommand{\FloatTok}[1]{\textcolor[rgb]{0.00,0.00,0.81}{#1}}
\newcommand{\FunctionTok}[1]{\textcolor[rgb]{0.00,0.00,0.00}{#1}}
\newcommand{\ImportTok}[1]{#1}
\newcommand{\InformationTok}[1]{\textcolor[rgb]{0.56,0.35,0.01}{\textbf{\textit{#1}}}}
\newcommand{\KeywordTok}[1]{\textcolor[rgb]{0.13,0.29,0.53}{\textbf{#1}}}
\newcommand{\NormalTok}[1]{#1}
\newcommand{\OperatorTok}[1]{\textcolor[rgb]{0.81,0.36,0.00}{\textbf{#1}}}
\newcommand{\OtherTok}[1]{\textcolor[rgb]{0.56,0.35,0.01}{#1}}
\newcommand{\PreprocessorTok}[1]{\textcolor[rgb]{0.56,0.35,0.01}{\textit{#1}}}
\newcommand{\RegionMarkerTok}[1]{#1}
\newcommand{\SpecialCharTok}[1]{\textcolor[rgb]{0.00,0.00,0.00}{#1}}
\newcommand{\SpecialStringTok}[1]{\textcolor[rgb]{0.31,0.60,0.02}{#1}}
\newcommand{\StringTok}[1]{\textcolor[rgb]{0.31,0.60,0.02}{#1}}
\newcommand{\VariableTok}[1]{\textcolor[rgb]{0.00,0.00,0.00}{#1}}
\newcommand{\VerbatimStringTok}[1]{\textcolor[rgb]{0.31,0.60,0.02}{#1}}
\newcommand{\WarningTok}[1]{\textcolor[rgb]{0.56,0.35,0.01}{\textbf{\textit{#1}}}}
\usepackage{graphicx,grffile}
\makeatletter
\def\maxwidth{\ifdim\Gin@nat@width>\linewidth\linewidth\else\Gin@nat@width\fi}
\def\maxheight{\ifdim\Gin@nat@height>\textheight\textheight\else\Gin@nat@height\fi}
\makeatother
% Scale images if necessary, so that they will not overflow the page
% margins by default, and it is still possible to overwrite the defaults
% using explicit options in \includegraphics[width, height, ...]{}
\setkeys{Gin}{width=\maxwidth,height=\maxheight,keepaspectratio}
\setlength{\emergencystretch}{3em}  % prevent overfull lines
\providecommand{\tightlist}{%
  \setlength{\itemsep}{0pt}\setlength{\parskip}{0pt}}
\setcounter{secnumdepth}{0}
% Redefines (sub)paragraphs to behave more like sections
\ifx\paragraph\undefined\else
\let\oldparagraph\paragraph
\renewcommand{\paragraph}[1]{\oldparagraph{#1}\mbox{}}
\fi
\ifx\subparagraph\undefined\else
\let\oldsubparagraph\subparagraph
\renewcommand{\subparagraph}[1]{\oldsubparagraph{#1}\mbox{}}
\fi

% set default figure placement to htbp
\makeatletter
\def\fps@figure{htbp}
\makeatother


\title{Statistical Inference Course Project - Part I}
\author{Evgeniia Golovina}
\date{12/03/2021}

\begin{document}
\maketitle

\hypertarget{overview}{%
\subsection{Overview}\label{overview}}

In this part of the project, we investigate the exponential distribution
and compare it with the Central Limit Theorem. The Central Limit Theorem
(CLT) is one of the most important theorems in statistics. The CLT
states that the distribution of averages of independent and identically
distributed (iid) variables becomes that of a standard normal as the
sample size increases.

Via simulation and associated explanatory text we aim to show the
following properties of the distribution of the mean of 40 exponentials:

\begin{enumerate}
\def\labelenumi{\arabic{enumi}.}
\tightlist
\item
  Show the sample mean and compare it to the theoretical mean of the
  distribution.
\item
  Show how variable the sample is (via variance) and compare it to the
  theoretical variance of the distribution.
\item
  Show that the distribution is approximately normal.
\end{enumerate}

\hypertarget{simulations}{%
\subsection{Simulations}\label{simulations}}

We set lambda equal to 0.2, number of eponential equal to 40 and number
of simulations equal to 1000.

\begin{Shaded}
\begin{Highlighting}[]
\CommentTok{# setting lambda (0.2), number of exponentials (40) amd number of simulations (1,000)}
\NormalTok{lambda <-}\StringTok{ }\FloatTok{0.2}\NormalTok{; n <-}\StringTok{ }\DecValTok{40}\NormalTok{; sim <-}\StringTok{ }\DecValTok{1000}

\KeywordTok{set.seed}\NormalTok{(}\DecValTok{1234}\NormalTok{)}
\CommentTok{# creating a data frame with 1000 X 40 from rexp(x, lambda) }
\NormalTok{sim_data <-}\StringTok{ }\KeywordTok{matrix}\NormalTok{(}\KeywordTok{rexp}\NormalTok{(n}\OperatorTok{*}\NormalTok{sim, lambda), }\DataTypeTok{nrow=}\NormalTok{sim, }\DataTypeTok{ncol=}\NormalTok{n)}
\CommentTok{# calculating the mean of each row (40 exponentials): sample mean}
\NormalTok{sim_means <-}\StringTok{ }\KeywordTok{apply}\NormalTok{(sim_data, }\DecValTok{1}\NormalTok{, mean)}
\end{Highlighting}
\end{Shaded}

\hypertarget{sample-mean-vs-theoretical-mean}{%
\paragraph{1. Sample mean vs theoretical
mean}\label{sample-mean-vs-theoretical-mean}}

First, we compare sample mean to the theoretical mean of the
distribution. The theoretical mean of the distribution is equal to
1/lambda.

\begin{Shaded}
\begin{Highlighting}[]
\CommentTok{# calculating sample mean}
\NormalTok{sample_mean <-}\StringTok{ }\KeywordTok{round}\NormalTok{(}\KeywordTok{mean}\NormalTok{(sim_means), }\DecValTok{3}\NormalTok{)}
\CommentTok{# calculating theoretical mean}
\NormalTok{theoretical_mean <-}\StringTok{ }\KeywordTok{round}\NormalTok{(}\DecValTok{1}\OperatorTok{/}\NormalTok{lambda, }\DecValTok{3}\NormalTok{)}
\CommentTok{# plotting mean distribution of 1000 simulations }
\KeywordTok{hist}\NormalTok{(sim_means, }\DataTypeTok{main=}\StringTok{"Histogram of 1000 means of 40 exponentials"}\NormalTok{, }\DataTypeTok{xlab=}\StringTok{"Sample mean"}\NormalTok{,}
     \DataTypeTok{ylab=}\StringTok{"Frequency"}\NormalTok{)}
\KeywordTok{abline}\NormalTok{(}\DataTypeTok{v=}\NormalTok{sample_mean, }\DataTypeTok{col=}\StringTok{"blue"}\NormalTok{, }\DataTypeTok{lwd=}\DecValTok{6}\NormalTok{)}
\KeywordTok{abline}\NormalTok{(}\DataTypeTok{v=}\NormalTok{theoretical_mean, }\DataTypeTok{col=}\StringTok{"red"}\NormalTok{, }\DataTypeTok{lwd=}\DecValTok{3}\NormalTok{)}
\end{Highlighting}
\end{Shaded}

\includegraphics{w4_pa_files/figure-latex/sample_mean_vs_theoretical_mean-1.pdf}

As we can see, sample mean (blue) of \emph{4.974} is very close to
theoretical mean (red) of \emph{5}.

\hypertarget{sample-variance-vs-theoretical-variance}{%
\paragraph{2. Sample variance vs theoretical
variance}\label{sample-variance-vs-theoretical-variance}}

Next, we show how variable the sample is (via variance) and compare it
to the theoretical variance of the distribution. The theoretical
standard deviation of the distribution is equal to 1/lambda.

\begin{Shaded}
\begin{Highlighting}[]
\CommentTok{# calculating the variance of the distribution}
\NormalTok{sample_var <-}\StringTok{ }\KeywordTok{round}\NormalTok{(}\KeywordTok{var}\NormalTok{(sim_means), }\DecValTok{3}\NormalTok{)}
\CommentTok{# calculating the theoretical variance of the distribution}
\NormalTok{theoretical_var <-}\StringTok{ }\KeywordTok{round}\NormalTok{((}\DecValTok{1}\OperatorTok{/}\NormalTok{lambda)}\OperatorTok{^}\DecValTok{2}\OperatorTok{/}\NormalTok{n, }\DecValTok{3}\NormalTok{)}

\CommentTok{# calculating sd of the sample mean}
\NormalTok{sample_sd <-}\StringTok{ }\KeywordTok{round}\NormalTok{(}\KeywordTok{sd}\NormalTok{(sim_means), }\DecValTok{3}\NormalTok{)}
\CommentTok{# calculating the theoretical sd}
\NormalTok{theoretical_sd <-}\StringTok{ }\KeywordTok{round}\NormalTok{((}\DecValTok{1}\OperatorTok{/}\NormalTok{lambda)}\OperatorTok{/}\KeywordTok{sqrt}\NormalTok{(n), }\DecValTok{3}\NormalTok{)}
\end{Highlighting}
\end{Shaded}

We can see that the sample variance of \emph{0.595} is very close to the
theoretical variance of \emph{0.625}. Sample standard deviation of
\emph{0.771} is also very close to the theoretical standard deviation of
\emph{0.791}.

\hypertarget{the-distribution-of-the-simulated-means}{%
\paragraph{3. The distribution of the simulated
means}\label{the-distribution-of-the-simulated-means}}

Lastly, we want to show that the distribution is approximately normal.

\begin{Shaded}
\begin{Highlighting}[]
\NormalTok{df <-}\StringTok{ }\KeywordTok{data.frame}\NormalTok{(sim_means); }\KeywordTok{names}\NormalTok{(df) <-}\StringTok{ }\KeywordTok{c}\NormalTok{(}\StringTok{"sim_mean"}\NormalTok{)}
\KeywordTok{ggplot}\NormalTok{(df, }\KeywordTok{aes}\NormalTok{(}\DataTypeTok{x=}\NormalTok{sim_mean)) }\OperatorTok{+}
\StringTok{       }\KeywordTok{labs}\NormalTok{(}\DataTypeTok{x=}\StringTok{"similuated mean"}\NormalTok{, }\DataTypeTok{title=}\StringTok{"Distribution of sample means vs theoretical mean"}\NormalTok{) }\OperatorTok{+}
\StringTok{       }\KeywordTok{geom_histogram}\NormalTok{(}\KeywordTok{aes}\NormalTok{(}\DataTypeTok{y=}\NormalTok{..density..), }\DataTypeTok{size=}\DecValTok{1}\NormalTok{, }\DataTypeTok{binwidth=}\FloatTok{0.2}\NormalTok{) }\OperatorTok{+}\StringTok{ }
\StringTok{       }\KeywordTok{geom_density}\NormalTok{(}\DataTypeTok{color=}\StringTok{"green"}\NormalTok{, }\DataTypeTok{size=}\DecValTok{1}\NormalTok{) }\OperatorTok{+}
\StringTok{       }\KeywordTok{geom_vline}\NormalTok{(}\DataTypeTok{xintercept=}\NormalTok{theoretical_mean, }\DataTypeTok{color=}\StringTok{"red"}\NormalTok{, }\DataTypeTok{size=}\DecValTok{1}\NormalTok{)}
\end{Highlighting}
\end{Shaded}

\includegraphics{w4_pa_files/figure-latex/normal_distribution-1.pdf}

The green line depicts the distribution of means of the simulated
samples. The red line is the theoretical mean. This figure shows that
the distribution of means of the simulated samples is very close to
normal distribution.

\end{document}
